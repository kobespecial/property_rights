%!TEX program = xelatex
% 完整编译: xelatex -> bibtex -> xelatex -> xelatex
\documentclass[lang=cn,12pt,a4paper]{elegantpaper}
% used packages 
\usepackage{graphicx}
\usepackage{slashed}
\usepackage{amsmath}
\usepackage{amstext}
\usepackage{amssymb}
\usepackage{amsthm}
\usepackage{ulem}
\usepackage{CJKfntef}
\usepackage{rotating} 
\usepackage{float}
\title{产权的XXXXX}
\author{汲铮 \and 梁寒 \and 卢遥}
\institute{}

\version{1.0}
\date{\zhtoday}

\begin{document}

\maketitle
参考文献

\begin{abstract}

\end{abstract}



\section{介绍}

\subsection{提要}

产权是人类文明社会迄今各种经济制度安排的基础,也是社会主义市场经济制度的基石。新近的研究表明,在人类史前漫长的采摘和狩猎岁月,乃至动物世界的某些物种中,已经表现出所有权争夺和防护现象。本文对迄今产权起源的先占先得说,劳动赋予说和秉赋效应说进行了梳理和剖析,指出秉赋效应作为产权自由交易实验的现象不能成为产生产权的前提。先占先得与劳动赋与具有同源性并提供了确立产权的物理基础,但对人身及其生存繁衍环境趋利避害的多重应激反应,才构成产权确立的生理、心理和社会基础。文章构建的演化博弈模型,证明了不需要资源争夺双方的主观估值差异,基于主体自我求生保全和繁衍的本能防卫需要,就可以更真实地演化出产权的初始和自然形态,但产权的稳定性和完整性有赖第三方力量的发育。同时,在这个过程中,人能力的异质性、资源禀赋差异和人类生存资料的成本收益变化还会不断塑造和影响产权制度的演变。文章对重新认识产权在人类社会中的作用和未来演变,具有启发性。

现代市场经济的本质是财产权利的交换,因此,财产权利的界定、确认和保护是市场经济的基石。中国自改革开放以来确立了社会主义市场经济的目标以来,产权问题上的任何风吹草动都会被认为是经济政策的风向标,关乎着经济社会和人心的稳定,所以产权的平等保护也一直是国家一再重申的经济法规和政策的重心所在。中国古代战国时期孟子就有“有恒产者有恒心”的判断。然而,产权制度回望过去确实又并非凝固不变,而是一个随着人类的认识能力、生产水平和经济社会发展而不断变迁的历史过程。随着现代生物学、生命科学、信息科学的不断突破性发展,人类对地球生物圈的进化过程及对现代人类的本质的认识,也在日新月异和空前地深化。产权与人类的关系究竟有多深,它从何而来,如何发生和演化,今后还会往哪里去,显然不仅对我们有强烈的现实意义,也有纵观历史的重要认识价值。本文在对迄今为止的文献进行梳理和剖析的基础上,提出自己的思路和认识,运用演化博弈模型推演产权的发端与演变,以期推动这个领域更深更广泛的讨论探索。


\subsection{引言}

《牛津法律大词典》定义产权“亦称财产所有权,是指存在于任何客体之中或之上的完全权利,它包括占有权、使用权、租借权、出借权、转让权、用尽权、消费权和其他与财产相关的权利”。当然也有很多人习惯于把所有权与产权包括的其他权利并列,并将所有权置于首位。中国在改革开放中被引进并产生了很大影响的是上个世纪以科斯为代表的现代产权理论,也被称为新制度主义和交易成本学派。当然,实际上产权制度及其解读至少与人类文明一样久远。古希腊哲学家亚里士多德就已经认识到,“人都爱自己,而自爱出于天赋,并不是偶发的冲动[人们对于自己的所有物感觉最好和快意;实际上是自爱的延伸]”。因为“人如果不具备必需的条件,他简直没法生活”。 \footnote{亚里士多德:《政治学》,吴寿彭译,第11页、55页,商务印书馆,2017年版。}故早在公元前五世纪制定的著名罗马《十二铜表法》中,所有权和占有就有一个单列的篇目。

与亚里士多德相同,霍布斯认为,人有运用自己的力量保存自己生命的自由,这就是人的自然权利。\footnote{霍布斯:《论公民》应星、冯克利译,贵州人民出版社,2003年版,第7页。}由于人自然本性和无休止的欲望,致使没有政府的自然状态是一种野蛮残暴的相互争夺的状态。\footnote{霍布斯:《利维坦》黎思复、黎延弼译,商务印书馆,2017年版,第94-96页。}同样,洛克认为,财产权是生命权与自由权的自然延伸,也是生命权与自由权的基础。自然权利是人天生就有的权利。人付出劳动即自然拥有财产权利。为了更好的保护生命、自由与财产,人们才让渡权利,订立契约,建立政府。\footnote{洛克:《政府论》赵伯英译,陕西人民出版社,2005年5月第1版,第144页。}康德强调了产权的排他性,他说``任何东西根据权利是我的,或者公正地是我的。由于它与我的关系如此密切,[以致]如果他人未曾得到我的同意而使用它,也就是对我的损害和冒犯''。\footnote{唐德:《法的形而上学原理------权利的科学》,沈叔平译,商务印书馆,1991年版第53页。}黑格尔重视产权定义中人的自由意志,``我作为自由意志在占有中成为我自己的对象——构成所有权的规定。'' “所有权是自由的最初定在”\footnote{黑格尔:《法哲学原理》,范扬译,商务印书馆,1961年,第54、59页。}

马克思吸取了古典哲学、古典经济学和早期社会主义思潮在产权研究上的认识成果,提出了马克思主义的产权学说,马克思认为,``一切生产都是个人在一定社会形式中并借这种社会形式进行的对自然的占有''。\footnote{《马克思恩格斯全集》第30卷,人民出版社,2004年,第28页。}``自己的劳动实际上是对自然产品的实际占有过程,''\footnote{《马克思恩格斯全集》第31卷,人民出版社,1998年,第349页。}
``分工发展的各个不同阶段,同时也就是所有制的各种不同形式''。``第一种所有制形式是部落【stamm】所有制。它与生产的不发达阶段相适应'',\footnote{《马克思恩格斯选集》第1卷,人民出版社,1995年版,第68页。}``劳动者``是自己的劳动能力、自己人身的自由所有者''。\footnote{《马克思恩格斯文集》第5卷,第195页。}``他以自己的劳动为基础的所有权'',\footnote{《马克思恩格斯全集》第31卷,第348页。}``构成个人的一切自由、活动和独立的基础''。\footnote{《马克思恩格斯文集》第2卷,人民出版社,2009年,第45页}马克思设想的未来社会是自由人的联合体,他们``在生产资料共同占有的基础上,重新建立个人所有制''。\footnote{《马克思恩格斯文集》第5卷,人民出版社,2009年,第874页}因此,人自身主体的自由与劳动和所有权的同一性\footnote{《马克思恩格斯文集》第30卷,第463页.}是马克思产权学说的重点。

当然,人类从自然状况走来,产权并非来自理论家的发明或先验的概念,而是人作为高级生物体生存竞争的产物。生物体及其基本组成单位的细胞,需要不断地和环境进行能量和物质的交换,才能维持新陈代谢从而生命本身。人类作为异养生物需要以其他生物及其有机合成物作为食物。因此对外部自然生存资料的率先乃至抢先取得、占有和消费行为是人作为生命体的本能。在漫长的史前时代,在与野兽及各种恶劣的自然环境共存的挣扎中,人们从先占先得的生存实践中发育出产权的意识、习俗和制度。所以,在人类有文字记载的文明史开始以后,我们在东西方都能看到先占先得在推动产权制度形成中的作用。在东方中国先秦时代就有先占获得所有权的法令,如秦简《田律》就明确规定了在官府允许范围内开垦、砍伐及渔猎的所有权。在西方先占确立所有权在法律上的系统规定可以追溯到古罗马法。古罗马法学家认为,``先占是自然法方式的典型代表''。``自然原则''、``自然能力''、``自然理性''要求无主之物归属最先占有者。先占取得的正当性在于``由于物是无主的,因而不会伤害任何人''。这也成为法学上所谓先占自由主义的来源。同时,罗马法学家也开始将物分为``可有物''与``不可有物'',无主物为``可有物'',对不可有物则不能先取得产权。这后来成为先占权主义的发端。现代多数国家实行将不动产实际归为不可有物,动产归为可有物的二元主义。应当指出,尽管在大多数国家的民法、行政法以及国际法中,先占原则都普遍存在。但先占确立产权迄今在我国法律中尚没有涉及。我国前几年出现的乌木、狗头金和陨石等所有权归属的争论,就与法律中先占原则及其应用范围的缺失有关。应当说,随着全球化的发展,无主的自然资源已经日渐稀少,但人的智力从而现代科学技术的迅猛发展,先占在知识产权和专利法领域,地位却日益重要。

现代产权是由法律界定的。因此,法律上的先占赋权与劳动赋与财产权利就成为一般解释产权来源的两个路径。\footnote{}但实际上更细致的考察可以发现,先占与劳动在产权的起源上其实不可分离。仅仅只是在大约一万年前,随着社会分工产生了畜牧业和农业,从而不动产成为产权的对象以及有组织的暴力可以实施财产占有控制之后,二者的分野才开始发生。在人类还从事采集和狩猎的漫长世纪中,先占与劳动是等同的。因为在采集和狩猎活动中,先占所不可缺少的谋划、寻找、发现、获取和支配消费过程,每一步都是劳动的付出。正如洛克所说:``人类一出生即享有生存权利,因而可以享用肉食和饮料以及自然所供应的以维持他们的生存的其他物品''。``每人对他自己的人身享有一种所有权,除了他以外任何人都没有这种权利。他的身体所从事的劳动和他的双手所进行的工作,我们可以说,是正当地属于他的。所以只要他使任何东西脱离自然所提供的和那个东西所处的状态,他就已经掺进他的劳动,在这上面参加他自己所有的某些东西,因而使它成为他的财产''。\footnote{洛克《政府论(下篇)》商务印书馆2018年版第17-18页。}有人也许会说先占是否包含更多运气的成份,其实劳动从来也是如此。你可能辛苦数日都没有发现一棵果树或可采摘到手的果实,也可能选择的路径正好发现了一片唾手可得的果树群。大自然并不总是根据人在每一个场合劳动付出的艰辛程度而给以回报。

上世纪70年代初,生物学家梅纳德·史密斯和普锐斯(Smith~and~Price1973)将博弈论应用于生物演化理论,指出在动物为争夺资源的竞争中存在着演化稳定性策略(ESS),并考察了一些动物在争夺资源中存在着在位占有者采取鹰派策略,对自己资源所有权的防卫现象。这样在资源双方实力相差不大时,资源争夺往往是以在位者的胜出而告终,这种所有权现象往往可以解释很多潜在而实际没有发生的竞争。\footnote{约翰·梅纳德·史密斯《演化与博弈论》,复旦大学出版社2016年版。}但该模型提示也可以存在反产权均衡。为避免这一点,Smith(1982)假设同一资源对于先占者和后来侵入者具有不同的价值,从而演化成先占者拥有产权、后来者就难以取得产权的非对称性。可以说此后产权起源问题的建模研究,均是沿着Smith这一赋予先占者、后来者对于物品不同的价值评估这一路径进行的。

Thaler(1980)后来在行为经济学研究中提出了禀赋效应,并陆续得到很多实验证明。禀赋效应是指,拥有某一物品的人对该物品的估值显著高于不拥有该物品的人的估值(Thaler,1980;Wilkinson和Klaes,2012)。Gintis(2007)将禀赋效应引入鹰鸽博弈模型用以解释产权的起源,即将Smith那里先占与后来者估值不同的假设变为可由禀赋效应证明的事实。``人们获得某物,似乎就立即获得了额外价值,这额外的价值仿若凭空而来,仅源于所有权这一事实''(Jonos\&Brosman,2008)。由于这种禀赋效应,那么先占者相对后来者会因估值高而投入更多成本去捍卫占有地位,这样产权就可以基于先占而建立(Gintis2009)。

Eswaran和Neary(2014)强调了先占和洛克的劳动赋与及其相互作用,内生和强化了二者在自然演化中对所有权诉求的心理乃至生物基础和实施能力。他们认为先占和劳动具有的实现产权控制的优势和意志,形成占有者高估值的非对称性,因此禀赋效应更象是进化史的遗痕。先占者获得的所有权有利于激励其进行生产性投资,进而最大化适存度,因而先占拥有产权是自然选择作用下演化的结果。不过,由于他们的模型仍然依赖占有者和后来者的估值差异形成的非对称性,因而难以区分其与禀赋效应模型的区别。

董志强和张永璟(2019)、Dong 和Zhang(2016)则是迄今沿着禀赋效应造就自然产权方向推进的最远和最坚决的,因而也有利于启发更多的思考。他们批评Eswaran和Neary援引先占和劳动赋于来论证产权形成的不稳固和不彻底性,认为禀赋效应是一种演化而来的个体固有的心理倾向,人们仅仅依赖物品所有权的抽象概念就可获得额外的价值,是自然产权形成的前提和基础。禀赋效应的高估占有物,充当了一种承诺机制,使得"愿意为捍卫拥有物而不惜代价"变得更加可信,因而这种看似错误的偏好反而改善了主体的处境,使禀赋效应得以成功演化。

不过,用禀赋效应去解释产权形成并非没有明显的缺陷。首先,禀赋效应是产权所有者自由交易时表现出的一种实验现象,这种效应受到交换物市场价格参照点的影响,表现的并不稳定,更多地表现为当事人有限理性的产物。在商品生产者和专业交易者那里,禀赋效应就趋向于消失。Apiceila等(2014)对非洲部落所做的考察实验也显示,与现代市场经济社会接触的渔猎采集部落里并不表现出禀赋效应,仅在与现代商业社会接触较多的部落里才有表现。如果这确是原始部落社会的普遍现象,用禀赋效应去解释产权起源就时空倒置了。

其次,也是更重要的,产权的确立首先要靠对抢夺和侵犯即强制的排除,有了这个前提才有自主交易。动物界的所有权现象全部来自于强力争斗而非交易的结果。强制和被迫与所有者在自由选择是否交易与以什么价格交易并从中解读出的禀赋效应在本质上是不同的。在确立产权后的自由交易中表现出的行为在逻辑上并不能作为产权形成的基础或前提。

最后,禀赋效应促生出产权,依赖于所有者对占有物偏离其真实价值的系统性高估。但如果在位者的高估就可以带来所有权的尊重进而产生适存性优势,那么争夺者在重复博奕的学习中迟早会明白这一点,从而去修正自己获胜后的估值预期,这样就会打破资源争夺双方的估值不对称。说由于占有者存在系统性的估值偏差能够使之更可信地捍卫所有权,反而有助于其适存性演化的解释,从最好处说也难免太牵强了。Sivan Frenkel et al (2018)为了解释高估值的行为偏见如何能成为进化中的优势,构建了禀赋效应与``赢者的诅咒''这两种偏见在长期可以相互纠正和低消的模型。但实际上,赢者的诅咒,仅仅是在拍卖和投标等特定情景中表现出来的一种高估交易对方货物价值的偏向,并不存在于日常的交易行为中。这两种根本不对称的偏见并不能在进化中相互抵消。总之归根结底,在原始丛林时代早期人类的自然产权形成,不会来自于各自尊重对方产权后的自由交易估值偏向,而是来自为何人们不是简单去抢夺别人手上而又自己心仪的物品?为何要开始就认可这是别人而不是大自然或自己的东西?

确实,产权的确立需要在位的占有者与意欲获取者之间存在不对称,否则争夺双方随意选择的鹰鸽策略可以导致各种博弈结果,但无法导致对占有或产权稳定的尊重。要做到这一点,其实我们并不需要禀赋效应给与所有者对占有物的人为高估值的假定,只需占有者对自己身体及生存繁衍依赖物存在保全防护反应。这是占有者所特有而非占有者没有的天然不对称。

生物学的研究表明,所有生物都要从外部捕获自由能来驱动化学反应,形成新陈代谢,才能维持生命活动。由于生物体内的新陈代谢所需要的物理、化学条件被限制在一个很窄的幅度内,生物体必须对外界产生影响生物体的变化、刺激,作出趋利避害的反应,以维持体内稳定和生命活动的应答,这种生物本能被称应激性。同时越是高等生物越是发展出多种感受刺激和作出反应的机制,从而在自然选择中适存和进化。求生保全和防护防卫反应就是动物更不必说是人类的主要应激反应机制。

生物学家的实验发现(Smith, 1982),如果两个阿拉伯雄性狒狒都认为自己是同一只雌性狒狒的主人的时候,持续和升级的战斗往往在两只雄性狒狒之间发生,但已经确立了所有权关系的雄性狒狒,则一般不会故意受到其他雄性的挑战。狮群中也有类似的情况。不过,一只雄狮在一个发情期是该雌狮的所有者的情况到下一个发情期中并不一定持续。Davies(1978)研究了林地上斑点树蝶之间竞争地盘的行为。雄性树蝶占据着日光的光斑,随太阳的移动而移动。占据光斑地盘者一般都会赶走入侵者,显示出所有权效应。但如果一只雄性个体被人为取走,直到有一个新的雄性个体占据了地盘,那么令人惊讶的是,老主人就会如一个新入侵者一样,不做多少努力就放弃争夺,新主人就能保持地盘的所有权。这似乎显示了占有的现时性。但当两只树碟同时触碰到植被时,双方的争夺可以是正常争夺时间的10倍。这与上述狒狒同时进入争夺的情形类似,都共同显示并不是对资源是否被自己占有的价值评估差异,而是争夺双方自认为谁是占有者的认知扮演了更决定性的作用。Beletsky\& Orians(1989)用美洲红翼鸫, Tobias(1997)用欧洲知更鸟做的动物实验显示,与上述的光斑树蝶不同,被一时被取走的动物返回时,这些老占有者并不会轻易放弃夺回资源的努力。这时新占有者占有资源的时间越长,越投入更大成本去战斗,从而越能成功捍卫所有权。在Krebs(1982)对大山雀的试验中,新占资源3小时的大山雀捍卫其占有地位比其作为新闯入者去挑战别人要多坚持7倍的时间,当新占10小时后对返回老主人的防卫战时间可多达14倍。这里反映的已经长时间不佔有的老主人其返回重夺资源的坚强意志,显然与禀赋效应所预测的相反。这表明不仅现时占有与否,而且占有的时间长度和历史、双方对自己占有身份的自信从而认知和情感因素在资源争夺和产权确立中占有重要地位。

人类作为地球上生物进化的万物之灵,更是发展出极为复杂的神经系统对他人与外部世界作出反应和应答的能力。Gintis例举蹒跚学步的孩子会随意去抓取和抢夺别人拥有的东西,但幼儿园的孩子却开始会尊重他人对物品的所有权,以说明禀赋效应的存在。实际上这与其说证明了幼儿园的孩子已经有了禀赋效应知道别人会比自己高估其拥有物而不去挑战,不如说这是因为幼儿园的孩子已经开始知道,正如他们会保护自己的东西一样,抢别人的东西会遇到抵抗和反对,故而抢夺一般并可取也不值得。这个例子似乎也表明保全防卫既有先天遗传的因素,也有后天习得的因素。因为动物的所有权防卫行为更多是一种遗传本能,其无抵抗地轻易放弃占有物会降低生存率和繁殖率,而人类则既有先天遗传本能又有后天文化传承和发展能力,这应当是为什么只有人类社会才发展出完整的产权意识和制度的原因。

本文第二节构建一个基本的演化模型,说明在佔有者为避免已佔资源的物质损失,免除被强制剥夺的挫折感和受辱感(斯密在《道德情操论》中说肉体感受到的痛苦一旦时过境迁就会被遗忘,而精神上的创伤则会长期缠绕挥之不去),和避免释放容易被欺负的博弈信号这三重效应的作用下,会自然投入对侵犯者的防卫作战。由于包括入侵者在内的每个人在变化了情况下也都必然是防卫者,因此这是公开信息。这种佔有者与入侵者成本收益的天然不对称,会在演化中形成对产权自发的尊重和维护。第三节在基本模型中引入博弈双方的能力实力差异以及信息不完全环境,证明在实力差异和信息不完全条件下,自然产权具有高度不稳定性。而原始人族群的适存性需要会在群体中内生出维护产权的第三方力量,维护族内产权秩序的族群会在遗传演化中人口繁衍壮大,进而增进族群的适存竞争力,从而在迭代演化中胜出。本文第四节进一步引入资源禀赋分配的差异和人类生存资料成本收益变化对演化博弈从而产权制度的影响,揭示产权制度演变的轨迹和可能方向。第五节为归纳和总结。



\subsection{文章结构	}

第二部分,我们从宏观层面的动态方程来分析种群演化过程和产权可能的产生;第三部分,我们从微观层面对个体进行Agent-Based模拟,来模拟基因层面的演化过程。


\section{模型}

\subsection{博弈形式}

根据演化博弈论的观点,在研究物种演化时,某类型物种的生存前景取决于其适存度(fitness),适存度由博弈过程中该个体得到的支付来描述,这里的类型(表现型)可看作该生物采取的策略。之所以被称为策略,是因为它是在研究动物行为中产生的,策略一词完全可以被“表现型”所代替。生物的任何一种表现型都可以被称为“策略”。

我们按照逻辑递进的顺序,从以下几种博弈形式来对产权的产生进行剖析。首先我们从最基本的对称博弈谈起,即博弈双方完全的地位相同。这种博弈如同现实世界中的同一物种中两个体的遭遇战,双方完全平等。然后我们会讨论非对称博弈,即双方地位不同,比如在位者和入侵者地位上的差异导致的策略集和收益的不同。然后我们会讨论多种群的情况,多种群显然也是非对称博弈的一种,但是由于来自一个种群的入侵者只会入侵另一种群,故和之前的非对称博弈有着本质的区别。

以上博弈均为个体无意识的,个体没有策略的选择,即不存在\textbf{策略更新},也没有所谓的博弈过程,只是简单的个体之间的遭遇,结果完全取决于双方的类型。对于有意识的个体,我们采取Brown(1951)提出的虚拟行动(fictitious play)方法,对于完全理性的个体,我们认为其使用最优反应策略。对于具有有限理性的物种,我们认为其使用有限最优反应策略(Fudenberg and Levine, 1998)。

我们还允许个体具有学习能力,存在\textbf{策略更新},其中包括了像高适存度个体学习,包括挑软柿子捏(信号效应)。我们还引入多期博弈中的适应性策略、竞争与合作,参与人在任何一期的策略取决于前一期对手的行为,并在此中引入道德惩罚。

如需要,我们还可以分析如下博弈形式
\begin{enumerate}
\item 模仿:每一期随机选出一名参与者,赋予其以一定概率模仿另一个随机参与人的策略的权力
\item 调整:整个种群应该朝更好的状态移动
\item 高维度博弈:每个个体的策略集包含$n>4$种策略
\end{enumerate}


\subsection{模型介绍}

演化博弈论最初为博弈论分支,对纳什均衡进行了精炼,精炼结果为演化稳定策略(ESS),即稀有变异下的稳定性,表现为不被偶然的变异物种所侵害的性质。现在已被越来越多地认为是一种自制动态系统,而在动力系统中,我们无需特别指明变异。

某种策略是否演化稳定?根据Maynard Smith and Price (1973),演化稳定表现为不被偶然的变异物种所侵害,即一个演化稳定策略ESS,如果整个种群中每个个体都采取这个策略,那么在自然选择下,不存在任何一种突变可以侵犯到这个种群,即不存在占优突变。只有演化稳定的类型才能在长期演化中存活下来。显然ESS是纳什均衡的一个精炼均衡。

我们先考虑一个较为简单的对称博弈$G=(A,A^{\bf{T}})$,其中$A=(\mu_{ij}),\ i,j\in\{1,\cdots,n\}$为支付矩阵,$\mu_{ij}$表示类型$i$的个体遇到类型$j$的个体时的支付。假设有$N$个参与人,构成参与人集$\mathscr{N}=\{1,2,\cdots,N\}$。我们允许参与人采取混合策略,则每个策略对应$n$维单纯形$S_{n-1}$中的一个点$\mathbf{p}$,即策略集
$$
\mathscr{S}=S_n=\{\mathbf{p}=(p_1,\cdots p_n)\in \mathbb{R} ^n :p_i\ge0,\Sigma_{i=1}^n p_i=1\}
$$
$S_{n-1}$的角为标准$n$维单位向量$\left\{\boldsymbol{e}_{1}, \boldsymbol{e}_{2}, \ldots, \boldsymbol{e}_{n}\right\}$(满足对$\forall i\in\{1,2,\cdots n\}$,有$\boldsymbol{e}_{ij}=\delta _{ij}$),分别对应$n$个纯策略。$S_{n-1}$的内部对应着所有的完全混合策略;对于$S_{n-1}$的边缘包含的策略,其支撑集定义了对应基底张成的边界面,即$\operatorname{supp}(\mathbf{p})=\left\{i: 1 \leq i \leq n \text  ,\quad p_{i}>0\right\}$为$\{1,2, \cdots ,n\}$的真子集。

类型为$\mathbf{p}$的个体遇到类型为$\mathbf{q}$的个体时的支付为$\mathbf{p}^{\bf{T}}A\mathbf{q}$\footnote{我们根据通常博弈论的假设,假设支付为线性函数,即使非线性,我们也可以采取局部线性化的方法来处理}。因此在此一般情形下,我们说一个策略$\mathbf{p}$为ESS,当且仅当该策略满足以下两条:

\begin{enumerate}
\item 纳什均衡条件:
\begin{equation}
\mathbf{p}^{\bf{T}}A\mathbf{p}\ge \mathbf{p}^{\bf{T}}A\mathbf{q}, \forall\mathbf{q}\in S_n
\end{equation}
\item 稳定性条件:
\begin{equation}
    \mathbf{p}^{\bf{T}}A\mathbf{p}=\mathbf{p}^{\bf{T}}A\mathbf{q},\mathbf{q}\ne \mathbf{p},\\
\mathbf{p}^{\bf{T}}A\mathbf{q} > \mathbf{q}^{\bf{T}}A\mathbf{q}
\end{equation}
\end{enumerate}
或者我们可采取另一种更为简便快捷的判定方法,根据(Hofbauer and Sigmund, 1998) 及(Cressman, 2003) ,一个策略$\mathbf{p}$为ESS,当且仅当该策略局部最优,即在$\mathbf{p}$的一个邻域内,满足
\begin{equation}
\mathbf{p}^{\bf{T}}A\mathbf{q}>\mathbf{q}^{\bf{T}}A\mathbf{q}, \text{for }\forall \mathbf{q}\ne\mathbf{p}
\end{equation}

若不考虑基因突变,演化博弈本质上就是常微分方程动力系统,于是我们可以根据动力系统的稳定性来描述产权的雏形。演化的动态可以通过在单纯形$S_{n-1}$上的微分方程来刻画,最主流的刻画来自于Taylor and Jonker (1978)提出的复制者动态(Replicator dynamics)。复制者动态为一组非线性、支付单调微分方程,不存在策略更新\footnote{$\sum {\dot {x_i}=0}$保证了单纯形结构为演化过程中的不变量。}。其中支付单调意味着$\frac{\dot{x}_{i}}{x_{i}}>\frac{\dot{x}_{j}}{x_{j}} \Longleftrightarrow u_{i}(\mathbf{x})>u_{j}(\mathbf{x})$,即高支付的类型比低支付的类型繁衍速度更快。




\subsection{对称博弈}

考虑对称博弈$G=(A,A^{\bf{T}})$,假设种群中所有个体可分为$n$类,丰度向量为$\mathbf{x}=(x_1,x_2,\cdots,x_n)$表示各类型的频次,满足$ \sum_{i=1}^nx_i=1$。对于个体$i$,其适存度函数为$f_{i}(\mathbf{x})=(A \mathbf{x})_{i}$。某类型个体数量的增加率取决于该类型个体的适存度同种群平均适存度的差,于是动态方程组为\footnote{此处我们采取复制者动态方程的Taylor形式,与之对应的还有Smith形式$\dot{{x}_{i}}=x_{i}\frac{\left((A \mathbf{x})_{i}-\mathbf{x}^{\bf{T}}  A \mathbf{x}\right)}{\mathbf{x}^{\bf{T}}  A \mathbf{x}}$,即演化的动力为相对适存度差}:
\begin{equation}
\dot{{x}_{i}}=x_{i}\left((A \mathbf{x})_{i}-\mathbf{x}^{\bf{T}}  A \mathbf{x}\right) \quad i=1, \ldots, n
\end{equation}
其中$\mathbf{x}^{\bf{T}}  A \mathbf{x} $为种群平均适存度,为此动力系统的\textit{势},也是系统的李雅普诺夫函数。根据演化博弈论的无名氏定理,所有ESS都是该系统的吸引子,满足李雅普诺夫渐进稳定,反之不一定成立(Hofbauer and Sigmund, 1998 & 2003)。对于非线性动力系统,我们通常采取的办法是通过局部直化定理和Hartman-Grobman定理来对其做局部线性化,具体过程参见附录中求解吸引子的部分。

对于遭遇战的鹰鸽二元博弈,本质上就是一个二维同质对称博弈,支付矩阵为
\begin{equation}
A=\begin{pmatrix}
\frac{1-c}{2} & 1\\
0 & \frac{1}{2}\\
\end{pmatrix}
\end{equation}
其中战斗胜利这适存度增加1(标准化),战斗成本为$c$。结果在word版本里,即根据战争的性质,ESS或为纯H,或为混合。

若存在报复者和欺软怕硬的类型,则……见word

\subsection{非对称博弈}

我们用非对称博弈来刻画那些处于不同地位的个体会有不同的策略集或不同的支付矩阵的情形。例如食物对于饥饿的个体比对吃饱的个体更有价值,战斗受伤风险对于强壮者更小,以及在位者和流浪者之间的天然不对称,在位者保具有先占优势,体现在哪怕自己是鸽,只要不被抢就仍占有资源。还比如在位者可能会拼死保护自己的资源,入侵者则往往虚晃一枪等不对称\footnote{我们这里并不排除反过来的情形,只是举个例子说明这种不对称的含义。}。因此对于非对称博弈,我们考察以下三种形式:1)双方地位不同导致策略集不同;2)双方策略集相同,但不同的地位有着不同的支付矩阵;3)地位的不同会影响策略集,或许也会影响支付。



我们考察一般化的双支付矩阵(bimatrix)博弈$G=(A,B^{\bf{T}})$,其中$\dim(A)=n,\ \dim (B)=m$,即个体1有n种策略,个体2有m种策略。Selten(1980)证明了在非对称博弈下一个策略为ESS当且仅当它是严格纳什均衡策略。在非对称博弈的不可压缩性表明内点都不是严格的,因此不存在混合策略ESS,所有ESS只可能出现在两单纯形的积空间的边界$\operatorname{bd} (S_{n-1} \times S_{m-1})$,均为纯策略(Cressman, 2003)。此时的动态方程为:
\begin{equation}
\dot{{x}_{i}}=x_{i}\left((A \mathbf{x})_{i}-\mathbf{x}^{\bf{T}}  A \mathbf{x}\right) \quad i=1, \ldots, n\\
\dot{{y}_{j}}=y_{j}\left((B \mathbf{y})_{j}-\mathbf{y}^{\bf{T}}  B \mathbf{y}\right) \quad j=1, \ldots, m
\end{equation}
积空间$ (S_{n-1} \times S_{m-1})$保持不变。支付矩阵和结果见word,原先对称时的ESS退化为两个纯策略均衡,即产权和反产权,其中的反产权能够解释自然界中某些蝴蝶种群的模式。

若存在报复者和欺软怕硬的类型,则……见word。

下文中出现的公式,如不加声明,我们总默认$\mathbf{x},\mathbf{y}$为个体采取的策略,$x_i,y_i$为频率,$A,B$为支付矩阵。

\subsubsection{角色博弈}



\subsubsection{多种群}

此时ESS未必是稳健的,考虑两个同质种群,分别为$\mathbf{p}$和$\mathbf{q}$策略选手,分别被一小部分$\mathbf{x},\mathbf{y}$入侵。x,y分别表示入侵种族的频率,动态方程为
\begin{equation}
\begin{aligned}
\dot{x} &=x(1-x)(b-(a+b) y) \\
\dot{y} &=y(1-y)(d-(c+d) x)
\end{aligned}
\end{equation}
其中\((a,b,c,d)=\)。必须同时满足$\mathbf{p}$不被$\mathbf{x}$侵害同时$\mathbf{q}$不被$\mathbf{y}$侵害的策略才是ESS。

\subsection{最优反应动态}

我们允许种群中的个体有偶尔的理性行为,我们用最优反应动态模型来刻画这一情况,即每一期会有小部分个体会根据最优反应函数来更新自己的策略,动态方程为
\begin{equation}
\dot{\mathbf{x}}=\mathrm{BR}(\mathbf{x})-\mathbf{x}
\end{equation}

通常解为线性轨道……相图为沙普利三角。任何复制者动态的内部ESS都是最优反应动态的全局渐进稳定点。

\subsection{有限理性动态}

对于低等生物,其个体往往不具备完全理性,因此我们赋予其偶尔的有限理性。我们用Fudenberg and Levine(1998)提出的Logit动态来刻画有限理性下的最优反应动态,动态方程为
\begin{equation}
\dot{{x}}_{i}=\frac{\exp \left[u_{i}(\mathbf{x}) / K\right]}{\sum_{j} \exp \left[u_{j}(\mathbf{x}) / K\right]}-{x}_i
\end{equation}
其中噪音参数$K$来衡量有限理性的程度,当$K\rightarrow0$时,退化到完全理性。

\subsection{适应性动态}






\section{模拟}

\subsection{Agent-Based Model}

之前的动态都是宏观层面的,从Agent-based开始涉及微观层面。Agent-Based动态往往由一个策略更新规则定义,这些规则可以用来模拟基因层面的演化过程,以及人类学习过程中的有限理性,两者都允许微小错误的存在。策略更新规则多种多样。

我们首先考虑一个同质参与者组成的系统,所有参与者构成一个格点架构(或图),处在x格点处的参与人采取某个纯策略,即从n个n维单位向量中选其一
\begin{equation}
\mathbf{s}_{x}=\left(\begin{array}{c}
1 \\
\vdots \\
0
\end{array}\right), \ldots,\left(\begin{array}{c}
0 \\
\vdots \\
1
\end{array}\right)
\end{equation}
其支付为
\begin{equation}
U_{x}=\sum_{y \in\mathcal{O}_{x}} \mathbf{s}_{x} ^{\bf{T}} \mathbf{A} \mathbf{s}_{y}
\end{equation}
其中$\mathcal{O}_x$是该参与人的“邻居”,由图的构架决定\footnote{这就是负能量状态下的(热力学)易辛模型,AB模型类似于这种简单的物理模型}。在此基础上,我们还可以继续按照更新策略的不同来对这种空间模型进行进一步的细分,这种细分有着明确的经济学和生物学上的意义。

\subsubsection{同时更新策略}

在空间模型中,是否同时更新策略对结果的影响很大。每一期(离散),所有参与者同时按照既定规则来更新自己的策略(Abramson and Kuperman, 2001),同时更新策略模型多被用在元胞自动机中,因此此种情况我们可用“平均场”元胞自动机来分析(晓齐),个体之间的战斗依照War of attrition规则进行。元胞自动机的结果通常分为四类,每一类可分别对应一种演化结果和产权特征。

进一步,我们可引入参与人的异质性和更新规则的随机性,这对于结果的影响可以产生一系列有建设性的结论。

\subsubsection{随机顺序更新策略}

这是一个非同时更新策略模型的最基本的特例。许多时候,种群中的每个个体更新策略的时间独立于其他个体,简化处理为每一期随机选出一个个体,只允许该个体更新策略。或者采取类似于北京市摇号系统规则的更公平的“泊松钟”模式,即久未中签的个体下一期中签的概率增加。

\subsubsection{微观更新规则}

更新策略的规则的核心参数之一为**个体转变率**$w(s\rightarrow s^\prime)$,表示在获得更新策略机会时,某个体从策略$s$更新为$s^\prime$的条件概率。假设每期随机允许一个个体更新策略,则任意个体必须满足$\sum_{s^\prime} w(s\rightarrow s^\prime)=1/n$。规则往往包括复制、模仿和学习。

\subsubsection{最优反应策略}

\subsubsection{胜者保持策略、败者转变策略}

若参与者的信息极度缺乏,以至于其余参与者的支付不可观察,此时参与者可采取这种策略,即胜者维持当前策略,负者随机改变策略。输赢的标准为前一期的支付是否高过某一心理预期值$a$(期望水平),这一规则在重复博弈中的应用即为著名的“巴甫洛夫规则”。






\section{结论}

\section{附录}






\nocite{*}
\bibliographystyle{IEEEtran.bst}
\bibliography{参考文献}

\end{document}
